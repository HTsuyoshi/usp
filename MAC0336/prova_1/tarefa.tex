%Baseado em https://www.overleaf.com/latex/templates/cse-3500-algorithms-and-complexity-homework-template/wrfwdhfzpnqc
\documentclass[12pt,letterpaper]{article}
\usepackage{fullpage}
\usepackage[top=2cm, bottom=4.5cm, left=2.5cm, right=2.5cm]{geometry}
\usepackage{amsmath,amsthm,amsfonts,amssymb,amscd}
\usepackage{lastpage}
\usepackage{enumerate}
\usepackage{fancyhdr}
\usepackage{mathrsfs}
\usepackage{xcolor} \usepackage{graphicx}
\usepackage{listings}
\usepackage{hyperref}
\usepackage{multicol}
\usepackage{xspace}

\usepackage[brazilian]{babel}

\hypersetup{%
  colorlinks=true,
  linkcolor=blue,
  linkbordercolor={0 0 1}
}

\setlength{\parindent}{0.0in}
\setlength{\parskip}{0.05in}

% Edit these as appropriate
\newcommand\course{MAC0336}
\newcommand\prof{Routo Terada}
\newcommand\hwnumber{1}                   % <-- homework number
\newcommand\NUSP{11796083}                % <-- NUSP
\newcommand\sname{Henrique Tsuyoshi Yara}               % <-- Name

\newcommand\answer{\textbf{Resolução.}\xspace}

\pagestyle{fancyplain}
\headheight 35pt
\lhead{\sname \\ \NUSP}
\chead{\textbf{\Large Prova \hwnumber}}
\rhead{\course\, - \prof \\ \today}       % \today deixa o dia de hoje automaticamente
\lfoot{}
\cfoot{}
\rfoot{\center\small\thepage}
\headsep 1.5em

\begin{document}

% Use \section*{} em vez de \section{} para evitar que o Latex numere as
% seções. Isso evita que fique redundante "1 Exercício 1", por exemplo.
\section*{Exercício 1}
\subsection*{1.A.1}
\answer \\
    \textbf{Parâmetros do} $Decripto$ \\

\begin{enumerate}
    \item $64$ é o número de bits em cada variável (i.e., cada posição de memória);
    \item $12$ é o número de iterações (rounds);
    \item $8$ é o número de bytes na chave.

\end{enumerate}

\subsection*{1.A.2}
\answer \\
    \textbf{Operações básicas}:

\begin{enumerate}
    \item $v \boxplus u$ é a soma dos inteiros v,u de 64 bits, resultando um valor de 64 bits ($i.e.$, soma $\mod$ 264);
    \item $v \oplus u$ é o ou-exclusivo (XOR) de v,u de 64 bits, resultando um valor de 64 bits;
    \item $v \gg t$ é o deslocamento circular (i.e., rotação) de t posições para a direita dos bits em v.
\end{enumerate}

\subsection*{1.A.3}
\answer \\
    \textbf{Algoritmo} Decripto \\
    \textbf{Entrada}: chave de 8 bytes, texto criptografado de $2 \times 64$ bits $(A, B)$; \\
    \textbf{Saída}: legível de $2 \times 64$ bits $(A, B)$; \\

\begin{enumerate}
    \item Calcular $2 \times 12 + 2$ subchaves $K_0, K_1, K_2,... K_{2 \times 12 + 1}$ (* Usando o algoritmo dado *)

    \item \textbf{para} $i = K_{2 \times 12 + 1}, K_{2 \times 12}, K_{2 \times 11 + 1},...K_{0}$: \\
        Achar $K_i^{-1}$, tal que \\
        $K_i \boxplus K_i^{-1} == 0$ \\
        $\overline{K_i} \leftarrow K_i^{-1}$

    \item \textbf{para} $j = 12, 11, 10,...1$ \textbf{faça}: \\
        $B \leftarrow ((A \boxplus \overline{K_{2j+1}}) \gg A) \oplus A$;$A \leftarrow ((A \boxplus \overline{K_{2j}}) \gg B) \oplus B$
    \item $B \leftarrow B \boxplus \overline{K_1}$;$A \leftarrow A \boxplus \overline{K_0}$
    \item A saída é o valor (A, B)
\end{enumerate}

\subsection*{1.B}
    Sendo $(A', B')$ os textos ilegíveis, $(A,B)$ os textos legíveis e $overline{K_0}, overline{K_1}, overline{K_2},...overline{K_{2 \times 12 + 1}}$ as subchaves complementares às subchaves $K_0, K_1, K_2,... K_{2 \times 12 + 1}$ em relação à $\boxplus$: \\
    Provando uma iteração:

\begin{enumerate}
    \item $B' = ((B \oplus A) \ll A) \boxplus K_{2_j + 1}$
    \item $B' \boxplus  \overline{K_{2_j + 1}} = (((B \oplus A) \ll A) \boxplus K_{2_j + 1}) \boxplus  \overline{K_{2_j + 1}}$
    \item $B' \boxplus \overline{K_{2_j + 1}} = ((B \oplus A) \ll A) $
    \item $(B' \boxplus \overline{K_{2_j + 1}}) \gg A = ((B \oplus A) \ll A) \gg A$
    \item $(B' \boxplus \overline{K_{2_j + 1}}) \gg A = (B \oplus A)$
    \item $((B' \boxplus \overline{K_{2_j + 1}}) \gg A) \oplus A = (B \oplus A) \oplus A$
    \item $((B' \boxplus \overline{K_{2_j + 1}}) \gg A) \oplus A = B$
\end{enumerate}

\begin{enumerate}
    \item $A' = ((A \oplus B) \ll B) \boxplus K_{2j}$
    \item $A' \boxplus \overline{K_{2j}} = (((A \oplus B) \ll B) \boxplus K_{2j}) \boxplus \overline{K_{2j}}$
    \item $A' \boxplus \overline{K_{2j}} = ((A \oplus B) \ll B)$
    \item $(A' \boxplus \overline{K_{2j}}) \gg B = ((A \oplus B) \ll B) \gg B$
    \item $(A' \boxplus \overline{K_{2j}}) \gg B = (A \oplus B)$
    \item $((A' \boxplus \overline{K_{2j}}) \gg B) \oplus B = (A \oplus B) \oplus B$
    \item $((A' \boxplus \overline{K_{2j}}) \gg B) \oplus B = A$
\end{enumerate}

    A operação inicial antes das iterações: \\
\begin{enumerate}
    \item $B' = B \boxplus K_1$
    \item $(B' \boxplus \overline{K_1}) = B \boxplus K_1 \boxplus \overline{K_1}) $
    \item $(B' \boxplus \overline{K_1}) = B \boxplus K_1$
\end{enumerate}

\begin{enumerate}
    \item $A' = A \boxplus K_0$
    \item $(A' \boxplus \overline{K_0}) = A \boxplus K_0 \boxplus \overline{K_0}) $
    \item $(A' \boxplus \overline{K_0}) = B \boxplus K_0$
\end{enumerate}
\qed

\subsection*{2}
\subsection*{2.1}
Calcular $T$ para $p = 7$, $S = 2$, e executar o Crip para $k = 3, x = 2$, obtendo $(y,z)$ \\
\answer \\
T = 2, y = 6, z = 2

\subsection*{2.2}
Justifique porque $k = 0$, e $k = 1$ devem ser evitados no Passo (1) do Crip. \\
\answer \\
    $K = 0$ deve ser evitado, pois $z = xT^0$ então $z = x$ o texto vai ser enviado sem estar criptografado. \\
    $K = 1$ deve ser evitado, pois $T$ é público e é fácil descobrir $T^{-1}$ e consequentemente descobrir $x$.

\subsection*{2.3}
\answer \\
\textbf{Algoritmo} Decriptografia do Crip \\
\textbf{Entrada} $(y, z)$ criptografados da alice \\

\begin{enumerate}
    \item $y \leftarrow y^{-1}$
    \item $y \leftarrow y^{S}$
    \item $x \leftarrow z \times y$
    \item A saída é o valor de $x$
\end{enumerate}

\subsection*{2.4}
\answer \\
    Executando o algoritmo obtenho $x = 2$.

\subsection*{2.5}
\answer \\

\begin{enumerate}
    \item $z \times y^-S \mod p$
    \item $xT^k \times y^-S \mod p$
    \item $x(g^S)^k \times y^-S \mod p$
    \item $x(g^S)^k \times (g^k)^-S \mod p$
    \item $x(g^S)^k \times g^{-kS} \mod p$
    \item $x(g^{Sk}) \times g^{-kS} \mod p$
    \item $xg^{Sk} \times g^{-kS} \mod p$
    \item $x \mod p$
\end{enumerate}

\subsection*{2.6}
\answer \\
    Ele é mais rápido, pois ele consiste em uma exponenciação e uma inversa, enquanto a criptogrfia consiste em duas exponenciações.

\subsection*{2.7}
\answer \\
    A definição do problema que faz essa criptografia computacionalmente segura é o Problema do Logaritmo Discreto. Se esse problema fosse de fácil solução o atacante poderia descobrir o valor de $k$ e assim descobrir o valor de $T^k$ e descobrir x com a equação $z(T^-k)$.

\subsection*{2.8}
\answer \\
    A razão de existir o NONCE k no Passo 1 é que caso o atacante consiga $x_1$ e $x_2$ que foram criptografados com um mesmo $k$, é possível obter $x_2$ usando $x_1$ na equação: \\
    $\frac{z_1}{z_2} = \frac{x_1}{x_2}$

\subsection*{2.9}
\answer \\

\subsection*{2.10}
\answer \\
    Autenticação do Rementente de Y é o destinatário ter a certeza de quem foi que mandou a mensagem. Ela não garante a autenticação do remetente, pois qualquer um pode forjar e se passar por Beto.

\subsection*{2.11}
\answer \\
    Autenticação do Destinatário de Y é apenas a pessoa que recebeu a informação consiga ler a mensagem. Ela garante a Autenticação do Destinatário pois, a chave secreta S é conhecida apenas pela Alice.

\subsection*{2.12}
\answer \\

\end{document}
