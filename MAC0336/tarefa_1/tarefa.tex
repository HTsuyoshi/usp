%Baseado em
% https://www.overleaf.com/latex/templates/cse-3500-algorithms-and-complexity-homework-template/wrfwdhfzpnqc
\documentclass[12pt,letterpaper]{article}
\usepackage{fullpage}
\usepackage[top=2cm, bottom=4.5cm, left=2.5cm, right=2.5cm]{geometry}
\usepackage{amsmath,amsthm,amsfonts,amssymb,amscd}
\usepackage{lastpage}
\usepackage{enumerate}
\usepackage{fancyhdr}
\usepackage{mathrsfs}
\usepackage{xcolor} \usepackage{graphicx}
\usepackage{listings}
\usepackage{hyperref}
\usepackage{multicol}
\usepackage{xspace}

\usepackage[brazilian]{babel}

\hypersetup{%
  colorlinks=true,
  linkcolor=blue,
  linkbordercolor={0 0 1}
}

\setlength{\parindent}{0.0in}
\setlength{\parskip}{0.05in}

% Edit these as appropriate
\newcommand\course{MAC0336}
\newcommand\prof{Routo Terada}
\newcommand\hwnumber{1}                   % <-- homework number
\newcommand\NUSP{11796083}                % <-- NUSP
\newcommand\sname{Henrique Tsuyoshi Yara}               % <-- Name

\newcommand\answer{\textbf{Resolução.}\xspace}

\pagestyle{fancyplain}
\headheight 35pt
\lhead{\sname \\ \NUSP}
\chead{\textbf{\Large Lista \hwnumber}}
\rhead{\course\, - \prof \\ \today}       % \today deixa o dia de hoje automaticamente
\lfoot{}
\cfoot{}
\rfoot{\center\small\thepage}
\headsep 1.5em

\begin{document}

% Use \section*{} em vez de \section{} para evitar que o Latex numere as
% seções. Isso evita que fique redundante "1 Exercício 1", por exemplo.
\section*{Exercício 1}

Uma fonte de informação $\mathcal{X}$ gera saídas $\{ x_1, \ldots, x_n\}$
com as probabilidades $p(x_1),$ $\ldots,$ e $p(x_n)$.
Lembre que a entropia de Shannon é definida como:
\[
    H \left( \mathcal{X} \right) = \sum_{i = 1}^{n} p(x_i) \log_2 \left( \frac{1}{p(x_i)} \right).
\]

%%%%%%%% ITEM 1.1 %%%%%%%%
\subsection*{1.1}
Escrever todos os passos do cálculo da entropia de $X$ para as seguintes probabilidades:
% Usei multicols aqui para os itens ocuparem menos espaço.
\begin{multicols}{3} % Poderia tirar essa linha
  \begin{itemize}
    \item $p(x_1) = 1/4$,
    \item $p(x_2) = 1/16$,
    \item $p(x_3) = 1/16$,
    \item $p(x_4) = 1/16$,
    \item $p(x_5) = 1/4$,
    \item $p(x_6) = 1/16$,
    \item $p(x_7) = 1/4$.
    \item[\vspace{\fill}] % Placeholder para ocupar as linhas que faltam quando
                          % os itens 7 são quebrados em colunas de 3.
  \end{itemize}
\end{multicols} % Poderia tirar essa linha

%%%%%%%% Resposta do item 1.1 %%%%%%%%

\answer \\
Usando a definição de entropia e aplicando os valores das probabilidades acima,
temos \ldots

\begin{enumerate}
    \item $E(X) = p(x_1)log_2[\frac{1}{p(x_1)}] + p(x_2)log_2[\frac{1}{p(x2)}] + p(x_3)log_2[\frac{1}{p(x_3)}] + p(x_4)log_2[\frac{1}{p(x_4)}] + p(x_5)log_2[\frac{1}{p(x_5)}] + p(x_6)log_2[\frac{1}{p(x_6)}] + p(x_7)log_2[\frac{1}{p(x_7)}]$
    \item $E(X) = \frac{1}{4} log_2[\frac{1}{(\frac{1}{4})}] + \frac{1}{16} log_2[\frac{1}{(\frac{1}{16})}] + \frac{1}{16} log_2[\frac{1}{(\frac{1}{16})}] + \frac{1}{16} log_2[\frac{1}{(\frac{1}{16})}] + \frac{1}{4} log_2[\frac{1}{(\frac{1}{4})}] + \frac{1}{16} log_2[\frac{1}{(\frac{1}{16})}] + \frac{1}{4} log_2[\frac{1}{(\frac{1}{4})}]$
    \item $E(X) = \frac{1}{4}log_2[4] + \frac{1}{16} log_2[16] + \frac{1}{16} log_2[16] + \frac{1}{16} log_2[16] + \frac{1}{4}log_2[4] + \frac{1}{16} log_2[16] + \frac{1}{4}log_2[4]$
    \item $E(X) = \frac{1}{4}*2 + \frac{1}{16} *4 + \frac{1}{16} *4 + \frac{1}{16} *4 + \frac{1}{4}*2 + \frac{1}{16} *4 + \frac{1}{4}*2$
    \item $E(X) = \frac{1}{2}*3 + 4*\frac{1}{4}$
    \item $E(X) = \frac{3}{2} + 1$
    \item $E(X) = \frac{5}{2}$
    \item $E(X) = 2,5$
\end{enumerate}

\qed % Desenha um quadradinho no fim para explicitar que acabou a resolução.
%%%%%%%% FIM DO ITEM 1.1 %%%%%%%%

\subsection*{1.2}
Para $j = 1,2,...n$, seja $max_j \lceil log_2[\frac{1}{p(x_j)}] \rceil = C$. Demostrar (i.e., provar matematicamente) que esse valor maximo C e o numero de bits dos $x_j$ : j = 1,2,...n. \\

\answer \\

Supondo que temos n indices na tabela o valor do indice precisa ser pelo menos $\lceil log_2(n) \rceil$. \\ \\

Para provar $C = max_j \lceil log_2[\frac{1}{p(x_j)}] \rceil \geq \lceil log_2(n) \rceil $. \\ \\

Sabendo que o menor $max_j$ possivel ocorre quando $p(x_i) = \frac{1}{n}$, pois para algum $p(x_i) < \frac{1}{n}$ vai existir $p(x_j) > \frac{1}{n}$ ja que $\sum_{i=1}^n p(x_i)= 1$ e o $max_j$ aumentaria.\\ \\

Portanto o menor $max_j \lceil p(x_i) \rceil$ possivel ocorre quando:\\ \\
$p(x_i) = \frac{1}{n}$

\begin{enumerate}
    \item $max_j \lceil log_2[\frac{1}{p(x_j)}] \geq \lceil log_2(\frac{1}{p(x_j)}) \rceil$
    \item $max_j \lceil log_2[\frac{1}{p(x_j)}] \geq \lceil log_2(\frac{1}{n^{-1}}) \rceil$
    \item $max_j \lceil log_2[\frac{1}{p(x_j)}] \geq \lceil log_2(n) \rceil$
\end{enumerate}

\qed

\subsection*{1.3}
Demonstrar que $log_2n$ é a entropia \textbf{máxima} de qualquer $X=\{x_1,x_2,...x_n\}$.\\ 
Supor dado o Lema: "A função $log_2()$ é estritamente côncava". E aplicar o Teorema de Jensen: "Se $f$: $\mathbb{R} \rightarrow \mathbb{R}$ é uma função contínua estritamente côncava no intervalo I, então $\sum_{i=1}^n a_i f (x_i) \leq  f ( \sum_{i=1}^n a_i x_i )$
\\
\answer

\begin{enumerate}
    \item Sendo a funçao $log_2()$ côncava, usando o teorema de jensen obtemos:
        \\ $\sum_{i=1}^n a_i log_2(x_i) \leq \sum_{i=1}^n a_i x_i$
    \item Relacionando com a formula da entropia:
        \\ $ a_i = p(x_i); x_i = p(x_i)^{-1}$
    \item $\sum_{i=1}^n p(x_i)log_2(p(x_i)^{-1}) \leq log_2(\sum_{i=1}^n p(x_i)p(x_i)^{-1})$
    \item $E(X) \leq log_2(\sum_{i=1}^n p(x_i)p(x_i)^{-1})$
    \item $E(X) \leq log_2(\sum_{i=1}^n 1)$
    \item $E(X) \leq log_2(n)$
\end{enumerate}

\qed

\subsection*{1.4}
Para qual conjunto $X$ essa entropia \textbf{máxima} ocorre? Demonstrar matematicamente esse fato. \\

\answer \\
Supondo um conjunto $X = \{x_1, x_2,... x_n\}$, sendo $\frac{1}{n} = p(x_1) = p(x_2) = ... = p(x_n)$, $\sum_{i=1}^np(x_i) = 1$ e $E(X) \leq log_2(n)$. Usando a formula da entropia:

\begin{enumerate}
    \item $E(X) = \sum_{i=1}^n \frac{1}{n} log_2(n)$
    \item $E(X) = \frac{n}{n} log_2(n)$
    \item $E(X) = log_2(n)$
\end{enumerate}

Portanto o conjunto $X$ tem a entropia máxima.

\qed


\section*{Exercício 2}

\subsection*{2.1.1}
Listar sua data de nascimento \\ 
\answer \\
29/07/01

\subsection*{2.1.2}
Listar o valor de $(E_0, D_0)$ em \textbf{hexadecimal}, como definimos anteriormente, para seu NUSP e sua data de nascimento.\\ \\
\answer \\
$(E_0, D_0): (0x11796022; 0x11796022)$ \\
$K  : 0x0209000700010000$

\subsection*{2.1.3}
Aceitar como entrada $(E_0,D_0)$ e a subchave $K_1$, e calcular e listar em \textbf{hexadecimal} a saída da primeira iteração (round 1), $(E1,D1)$. conforme o desenho dado de 1 iteração (round). A subchave deve ser gerada com a chave $K$ definida com os seus dados. \\ 
\answer \\
$(E_1, D_1): (0x00ee2288; 0x32a6f73e)$ \\
$K_1  : 0x000000010009$

\subsection*{2.1.4}
Complementar apenas o bit mais à esquerda de E0 e calcular e listar em \textbf{hexadecimal} a saída da primeira iteração (round 1), $(E_1^c ,D_1^c)$ \\
\answer \\ 
$(E_0^c, D_0): (0x91796022; 0x11796022)$\\
$(E_1, D_1): (0x01ee2288; 0x76aff62e)$\\
$K_1  : 0x000000010009$

\subsection*{2.1.5}
\textbf{Calcular} e listar o número de bits diferentes entre $(E_1,D_1)$ e $(E_1^c ,D_1^c)$ \\
\answer \\

A diferença entre $(E_1, D_1)$ e $(E_1^c, D_1^c)$ é de 7 bits

\subsection*{2.2.1}
Efetuar os mesmo passos (2) a (5) para cada iteração $j = 2,3,4,..16$, ou seja, calcular e listar o número de bits diferentes entre $E_j,D_j$e $E_j^c,D_j^c$

\answer

\begin{center}
\begin{tabular}{||c|c|c|c|c|c|c||} 
 \hline
    $Round_j$ & subch. $K_j$ & $E_j$ & $E_j^c$ & $D_j$ & $D_j^c$ & bits diff \\ 
 \hline
    0 & 0x000000000000 & 0x11796022 & 0x91796022 & 0x11796022 & 0x11796022 & 1\\
 \hline
    1 & 0x000000010009 & 0x00ee2288 & 0x01ee2288 & 0x32a6f73e & 0x76aff62e & 7\\
 \hline
    2 & 0x000000200680 & 0x32a6f73e & 0x76aff62e & 0x63c30f2a & 0xbd7b26f0 & 24\\
 \hline
    3 & 0x000000180003 & 0x63c30f2a & 0xbd7b26f0 & 0xcc519dad & 0x286c8ce8 & 32\\
 \hline
    4 & 0x000000064000 & 0xcc519dad & 0x286c8ce8 & 0xde091d5e & 0x2470e4f7 & 35\\
 \hline
    5 & 0x000000002140 & 0xde091d5e & 0x2470e4f7 & 0x8902388e & 0xffbb0a71 & 42\\
 \hline
    6 & 0x000000a08000 & 0x8902388e & 0xffbb0a71 & 0xb3fc4207 & 0x22ac1cb8 & 38\\
 \hline
    7 & 0x000000400602 & 0xb3fc4207 & 0x22ac1cb8 & 0x3a35e3ac & 0x7b588e0a & 33\\
 \hline
    8 & 0x0000001c0008 & 0x3a35e3ac & 0x7b588e0a & 0x3bd3c313 & 0xcdb83d43 & 36\\
 \hline
    9 & 0x000000000424 & 0x3bd3c313 & 0xcdb83d43 & 0x284bd431 & 0xee94f446 & 38\\
 \hline
    10 & 0x000000480880 & 0x284bd431 & 0xee94f446 & 0x34799fd6 & 0xaad372a1 & 39\\
 \hline
    11 & 0x000000004019 & 0x34799fd6 & 0xaad372a1 & 0x4357cdb7 & 0xed95b64d & 41\\
 \hline
    12 & 0x000000031000 & 0x4357cdb7 & 0xed95b64d & 0x61f21e01 & 0x06ed9081 & 35\\
 \hline
    13 & 0x000000800120 & 0x61f21e01 & 0x06ed9081 & 0xde7ca3e3 & 0xc3da3f00 & 32\\
 \hline
    14 & 0x000000000a04 & 0xde7ca3e3 & 0xc3da3f00 & 0x0bd007e4 & 0xc9985cba & 32\\
 \hline
    15 & 0x000000500090 & 0x0bd007e4 & 0xc9985cba & 0xbe348e0f & 0x4d2d2eee & 30\\
 \hline
    16 & 0x00000080a004 & 0xbe348e0f & 0x4d2d2eee & 0x41078409 & 0xdd39c753 & 31\\
 \hline
\end{tabular}
\end{center}

\subsection*{2.3.1}
Listar $K$ em \textbf{hexadecimal} \\ 
\answer \\ \\
$K = 0x0209000700010000$

\subsection*{2.3.2}
Aceitar como entrada $(E_0, D_0)$, como definido anteriormente, listar esses valores, e calcular a saída da primreira iteração $(round 1),(E_1, D_1)$ \\ \\ 
\answer \\ \\
$(E_0, D_0): (0x11796022; 0x11796022)$ \\
$K_1  : 0x000000010009$ \\
$(E_1, D_1): (0x00ee2288; 0x32a6f73e)$ \\
$K_1  : 0x000000010009$

\subsection*{2.3.3}
Listar os valores da subchave $K_1, (E_1, D_1)$ em \textbf{hexadecimal} \\
\answer \\ \\
$(E_1, D_1): (0x00ee2288; 0x32a6f73e)$ \\
$K_1  : 0x000000010009$

\subsection*{2.3.4}
Complementar apenas o bit mais à esquerda da \textbf{chave} $K$ (Sem alterar a entrada), e listar esse valor em \textbf{hexadecimal} \\ 
\answer \\ \\
$K = 0x8209000700010000$

\subsection*{2.3.5}
Calcular a subchave $K_1^c$ e a saída da primeira iteração (round 1), $(E_1^c, D_1^c)$ \\ \\
\answer \\ \\
$(E_1^c, D_1^c): (0x00ee2288; 0xb2a6e77e)$ \\
$K_1^c  : 0x000010010009$

\subsection*{2.3.6}
Listar os valores $K_1^c, (E_1^c, D_1^c)$ em \textbf{hexadecimal} \\
\answer \\ \\
$(E_1^c, D_1^c): (0x00ee2288; 0xb2a6e77e)$ \\
$K_1^c  : 0x000010010009$

\subsection*{2.3.6}
Listar os valores $K_1^c, (E_1^c, D_1^c)$ em \textbf{hexadecimal} \\
\answer \\ \\
A diferença entre $(E_1, D_1)$ e $(E_1^c, D_1^c)$ é de 3 bits

\subsection*{2.4.1}
Efetuar os mesmos passos (2) a (7) para cada iteração $j = 2,3,4,..16$, ou seja, calcular e listar $K_j$ em hexadecimal e o número de bits diferentes entre $(E_j,D_j)e (E_j^c,D_j^c)$ \\
\answer

\begin{footnotesize}
\begin{center}
\begin{tabular}{||c|c|c|c|c|c|c|c||} 
 \hline
    $R_j$ & $K_j$ & $K_j^c$ & $E_j$ & $E_j^c$ & $D_j$ & $D_j^c$ & bits diff \\ 
 \hline
    1 & 0x000000010009 & 0x000010010009 & 0x00ee2288 & 0x00ee2288 & 0x32a6f73e & 0xb2a6e77e & 3\\
 \hline
    2 & 0x000000200680 & 0x004000200680 & 0x32a6f73e & 0xb2a6e77e & 0x63c30f2a & 0x3065e338 & 18\\
 \hline
    3 & 0x000000180003 & 0x000100180003 & 0x63c30f2a & 0x3065e338 & 0xcc519dad & 0x0ac7d791 & 30\\
 \hline
    4 & 0x000000064000 & 0x000001064000 & 0xcc519dad & 0x0ac7d791 & 0xde091d5e & 0x2eb32c77 & 30\\
 \hline
    5 & 0x000000002140 & 0x010000002140 & 0xde091d5e & 0x2eb32c77 & 0x8902388e & 0x0d68f79c & 29\\
 \hline
    6 & 0x000000a08000 & 0x000080a08000 & 0x8902388e & 0x0d68f79c & 0xb3fc4207 & 0x10a4345b & 30\\
 \hline
    7 & 0x000000400602 & 0x100000400602 & 0xb3fc4207 & 0x10a4345b & 0x3a35e3ac & 0x26c7b20c & 29\\
 \hline
    8 & 0x0000001c0008 & 0x0000001c0008 & 0x3a35e3ac & 0x26c7b20c & 0x3bd3c313 & 0x8e5a95d4 & 30\\
 \hline
    9 & 0x000000000424 & 0x002000000424 & 0x3bd3c313 & 0x8e5a95d4 & 0x284bd431 & 0xc4488c24 & 30\\
 \hline
    10 & 0x000000480880 & 0x000400480880 & 0x284bd431 & 0xc4488c24 & 0x34799fd6 & 0xf569044f & 26\\
 \hline
    11 & 0x000000004019 & 0x400000004019 & 0x34799fd6 & 0xf569044f & 0x4357cdb7 & 0xc4bc8f1e & 29\\
 \hline
    12 & 0x000000031000 & 0x008000031000 & 0x4357cdb7 & 0xc4bc8f1e & 0x61f21e01 & 0x874ed128 & 35\\
 \hline
    13 & 0x000000800120 & 0x000002800120 & 0x61f21e01 & 0x874ed128 & 0xde7ca3e3 & 0x7c8f81e0 & 32\\
 \hline
    14 & 0x000000000a04 & 0x200000000a04 & 0xde7ca3e3 & 0x7c8f81e0 & 0x0bd007e4 & 0x3c6a71e5 & 29\\
 \hline
    15 & 0x000000500090 & 0x000000500090 & 0x0bd007e4 & 0x3c6a71e5 & 0xbe348e0f & 0x7ca6e7a4 & 31\\
 \hline
    16 & 0x00000080a004 & 0x00004080a004 & 0xbe348e0f & 0x7ca6e7a4 & 0x41078409 & 0x30797def & 36\\
\hline
\end{tabular}
\end{center}
\end{footnotesize}

\subsection*{2.5.1}
Supondo a chave $K$ permaneça fixa, o resultado numérico que V obteve indica algum nível de dificuldade de um mal-intencionado calcular a entrada $(E_0,D_0)$ corresponde a uma dada saída de 64 bits? Por que? \\

\answer \\
Supondo que a chave $K$ permaneça fixa V vai obter vários resultados numéricos criptografados por uma mesma chave, o problema que isso causa é que se V testar a mesma chave em várias mensagens ele vai conseguir decriptografar todas.

\subsection*{2.5.2}
A mesma pergunta, se fosse só uma iteração? \\

\answer \\
Sendo apenas uma iteração, a mensagem criptografada vai estar mais vulnerável ainda. O nível de entropia vai estar bem baixa, consequentemente a difusão e a confusão também, então duas mensagens criptografada com a mesma chave vão estar bem parecidas.

\subsection*{2.5.3}
Supondo que a entrada $(E_0,D_0)$ permaneça fixa, o resultado numérico que V obteve indica algum nível de dificuldade de um mal-intencionado calcular a chave $K$ correspondente a uma dada saída de 64 bits? Por quê? \\

\answer \\
Supondo que a entrada $(E_0,D_0)$ permaneça fixa V vai obter vários resultados numéricos criptografados por chaves diferentes, o problema que isso causa é que várias chaves vão conseguir decriptografar a mensagem.

\subsection*{2.5.4}
A mesma pergunta, se fosse só uma iteração? \\

\answer \\
Sendo apenas uma iteração, a mensagem criptografada vai estar mais vulnerável já que nível de entropia vai estar bem baixa então uma mensagem criptografada com chaves diferentes vão estar bem parecidas.

\subsection*{2.5.5}
Qual a relação do resultado numérico que V obteve com o conceito de Entropia de Informação segundo Shannon? \\ 

\answer \\ 
A relação que foi obtida é que a entropia (confusão e difusão) tem uma grande importância em proteger as informações das mensagens e das chaves

\end{document}
