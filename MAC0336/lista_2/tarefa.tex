%Baseado em
% https://www.overleaf.com/latex/templates/cse-3500-algorithms-and-complexity-homework-template/wrfwdhfzpnqc
\documentclass[12pt,letterpaper]{article}
\usepackage{fullpage}
\usepackage[top=2cm, bottom=4.5cm, left=2.5cm, right=2.5cm]{geometry}
\usepackage{amsmath,amsthm,amsfonts,amssymb,amscd}
\usepackage{lastpage}
\usepackage{enumerate}
\usepackage{fancyhdr}
\usepackage{mathrsfs} \usepackage{xcolor} \usepackage{graphicx} \usepackage{listings}
\usepackage{hyperref}
\usepackage{multicol}
\usepackage{xspace}

\usepackage[brazilian]{babel}

\hypersetup{%
  colorlinks=true,
  linkcolor=blue,
  linkbordercolor={0 0 1}
}

\setlength{\parindent}{0.0in}
\setlength{\parskip}{0.05in}

% Edit these as appropriate
\newcommand\course{MAC0336}
\newcommand\prof{Routo Terada}
\newcommand\hwnumber{1}                   % <-- homework number
\newcommand\NUSP{11796083}                % <-- NUSP
\newcommand\sname{Henrique Tsuyoshi Yara}               % <-- Name

\newcommand\answer{\textbf{Resolução.}\xspace}

\pagestyle{fancyplain}
\headheight 35pt
\lhead{\sname \\ \NUSP}
\chead{\textbf{\Large Lista \hwnumber}}
\rhead{\course\, - \prof \\ \today}       % \today deixa o dia de hoje automaticamente
\lfoot{}
\cfoot{}
\rfoot{\center\small\thepage}
\headsep 1.5em

\begin{document}

% Use \section*{} em vez de \section{} para evitar que o Latex numere as
% seções. Isso evita que fique redundante "1 Exercício 1", por exemplo.
\section*{Exercício 1}

\subsection*{1.1}
Implementar em linguagem Python o algoritmo de exponenciação modular descrito abaixo \\

\answer \\
Está dentro do arquivo compactado.

\subsection*{1.2}
Listar o seu número USP de 6 dígitos. \\

\answer \\
$117960$

\subsection*{1.3}
Calcular e listar seu NUSP elevado a $2345$, $\mod 6789$. E.g., para esse NUSP: $298765^{2345} \mod 6789 = 52326$ Sugestão: fazer testes com números de 2 ou 3 dígitos \\

\answer \\
$117960^{2345} \mod 6789 = 1761$

\subsection*{1.4}
Justificar a complexidade de tempo de execução deste algoritmo que é O(log e). \\

\answer \\
Considerando o números de atribuições e comparações.
Sabendo que $t = \lfloor log_2(e) \rfloor + 1$ é o número de bits do número $e$. \\

Na linha 1: \\
$\theta(1)$ \\

No laço de t até 0: \\
$\theta(\lfloor log_2(e) \rfloor + 1) \times (\theta(1) + O(1))$ \\

Na última linha: \\
$\theta(1)$ \\

Temos a equação: \\
$f(e) = \theta(2) + O((1 + log_2(e)) \times 2)$ \\
$f(e) = \theta(2) + O(2 + log_2(e) \times 2)$ \\
$f(e) = O(4 + log_2(e) \times 2)$ \\

Desconsiderando os valores constantes temos o tempo de complexidade: \\
$O(log_2(e))$

\qed

\section*{Exercício 2}

\subsection*{2.1}
Com parâmetro $5 \leq w \leq 10$, calcular um número inteiro primo absoluto maior que o inteiro correspondente aos 8 dígitos da sua data de nascimento $ddmmaaaa$. E.g., para a data 22/10/8899, os dígitos são 22108899. E 22109777, 221101061 são primos absolutos. Sugestão: fazer testes com números de 2 ou 3 dígitos. \\

\answer \\
Minha data: 29072001 \\
O número inteiro primo absoluto é 29072009 \\

\subsection*{2.2}
Deduzir e justificar o tempo de execução deste algoritmo como função de $w$ e de $y$. \\

\answer \\
Considerando o número de atribuições e comparações. \\
Sendo que a função $g(n-1)$ calcula $t$ e $c$, e o tempo de complexidade é o número de zeros entre o bit menos significate (inclusivo) até o primeiro bit 1 (exclusivo) $g(n-1) = O(\lfloor \log_2 \left[ n-1 \right] \rfloor)$: \\

Para obter t e c: \\
$g(y-1)$ \\

\begin{tabbing}
No \= loop de 1 até w: \\
\> Escolher um testemunho: \\
\> $\theta(1)$ \\ 

\> Atribuir valor à $r_0$ e $r_1$: \\
\> $\theta(2)$ \\

\> No \= loop de 1 até t: \\

\> \> Comparar valores de $r_j$ e $r_{j-1}$: \\
\> \> $O(3)$ \\ \\

\> \> Atribuir valor de $r_j+1$: \\
\> \> $\theta(1)$ \\
\end{tabbing}

Temos a equação: \\
$f(y, w) = g(y-1) + O(w) \times (\theta(1) + \theta(2) + O(\lfloor \log_2 \left[ y-1 \right] \rfloor) \times (O(3) + \theta(1)))$ \\
$f(y, w) = O(\lfloor \log_2 \left[ y-1 \right] \rfloor) + O(w) \times (\theta(3) + O(\log_2 \left[ y-1 \right] ) \times O(4))$ \\
$f(y, w) = O(\log_2 \left[ y-1 \right]) + O(3w) + O(4w) \times O(\log_2 \left[ y-1 \right] )$ \\

Desconsiderando os valores constantes temos o tempo de complexidade: \\
$f(y, w) = O(w)O(\log_2 \left[ y-1 \right] ) + O(\log_2 \left[ y-1 \right]) + O(w)$ \\


\section*{Exercício 3}

\subsection*{3.1}
Calcular e listar dois inteiros primos absolutos $q,r$, não necessariamente consecutivos, cada um maior que o inteiro correspondente aos 8 dígitos da sua data de nascimento ddmmaaaa. E.g., para a data 22/10/8899, os dígitos são 22108899. \\

\answer \\
$q = 29072881; r = 29072921$

\subsection*{3.2}
Calcular e listar $n = q \times r$. \\
E.g., $22109776 \times 221101061 = 4888495153173397$ \\

\answer \\
$n = q \times r = 29072921 \times 29072881 = 845233572555401$

\subsection*{3.3}
Calcular e listar $\phi(n) = (q - 1)(r - 1)$(Função de Euler). E.g., $22109776 \times 221101060 = 4888494909962560$ \\

\answer \\
$\phi = (q-1)(r-1) = 29072920 \times 29072880 = 845233514409600$

\subsection*{3.4}
Sortear aleatoriamente e listar a chave secreta sdo RSA com pelo menos 10 dígitos. E.g., $s = 1234567899$ \\

\answer \\
$s = 7290335707$

\subsection*{3.5}
Calcular e listar a chave pública pcorrespondente a $s: p \times s = 1 \mod(n)$. \\
E.g., $1234567899^{-1} \mod 4888494909962560$ = $3858608707133139$ \\

\answer \\
$p = 310717853942428 , s = 7290335707,310717853942428 \times 7290335707 \mod 845233514409600 = 1$

\subsection*{3.6}
Calcular e listar $p \times s \mod \phi(n) = ps$.\\
E.g., $1234567899 \times 3858608707133139 \mod 4888494909962560 = 1$ \\

\answer \\
$p = 81213405628243, s = 7290335707, 81213405628243 \times 7290335707 \mod 845233514409600 = 1$

\subsection*{3.7}
Seja $x_0$ o seu NUSP repetido várias vezes para completar pelo menos 15 dígitos. \\ 
E.g., para o NUSP 298765, os 15 dígitos são $298765298765298 = x_0$ \\

\answer \\
$x_0 = 117960117960117$

\subsection*{3.8}
Listar $x_0$ = $x_0$ \\

\answer \\
$x_0 = 117960117960117$

\subsection*{3.9}
Calcular e listar $y_0 = RSA(x_0,p) = (x_0)^p \mod n$. \\
E.g., $298765298765298^{3858608707133139} \mod 4888495153173397$ = $180585179472907$ \\

\answer \\
$y_0 = 55044909303457$

\subsection*{3.10}
Calcular e listar $x_1 = RSA^{-1}(y_0)$. \\
E.g., $180585179472907^{1234567899} \mod 4888495153173397 = 298765298765298$ \\

\answer \\
$x_1 = 117960117960117$

\subsection*{3.11}
Calcular e listar $RSA(x_1)$. \\
E.g., $298765298765298^{3858608707133139} \mod 4888495153173397 = 180585179472907$ \\

\answer \\
$y_1 = 55044909303457$

\end{document}
